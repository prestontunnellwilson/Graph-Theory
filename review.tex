\documentclass{article}
\usepackage{amssymb}
\usepackage{amsfonts}
\usepackage[margin = 1in]{geometry}

\begin{document}

\title{
Review
}

\author{
Preston Tunnell Wilson }

\date{
October 8, 2014 }

\maketitle

\section{Overview of Graphs}
\subsection{Definition}
A $graph$ $G$ is defined as a set of vertices $V$ and a set of edges $E$.
Note that $E$ is actually a multiset-duplicate elements are allowed.

A $loop$ is a vertex connected to itself by one edge.

A $simple \ graph$ has no loops and no multiple edges.

The $degree$ of a vertex is the number of edges coming into it or leaving it.

A $complete \ graph$ is one in which every two distinct vertices are adjacent.
A complete graph on $n$ vertices is denoted $K_n$.

A $connected \ graph$ is one such that there exists a walk from every two vertices.

The number of edges on a complete graph is $\frac{n(n-1)}{2} = {n \choose 2}$.
The sum of degrees is two times the number of edges.

\subsection{Isomorphy}
Two graphs being $isomorphic$ is like being equivalent but in a graph theory sense.

$f$ is an $isomorphy$ if $f$ is a bijection $V \rightarrow V'$ and $\forall x, y \in V,
(x, y) \in E <=> (f(x), f(y)) \in E'$.
Basically, you can remap every vertex to a corresponding one in the other graph
while maintaining edge relationships.

\subsection{Traversing a Graph}
A $walk$ is a set of vertices $\lbrace v_0, v_1, ... \rbrace$ such that $(v_{i-1}, v_i) \in E$.

A $tour$ is a walk with $v_0 = v_i$.

A $path$ is a walk such that each vertex appears at most once.

A $cycle$ is a closed path such that $v_0 = v_i$. 
Note that a loop counts as a cycle.

A graph is connected if there exists a walk between every $2$ vertices.

A $subgraph$ is defined as $G'=(V', E')$, where $V'\subset V$ and $E'\subset E$: 
$e\in E' => a,b \ in V', (a, b) = e$.

An $induced \ subgraph$ is defined as a subset of the vertices and all edges between those vertices.

A $component$ is a maximum induced subgraph, where maximum means that we cannot add any more vertices
and remain induced.

A $tree$ is a connected graph with no cycles. A $leaf$ is a vertex on a tree with degree $1$.

\subparagraph{Eulerian junk}
A $trail$ is a walk that travels each edge $\leq 1$ time.

An $Euler \ trail$ of $G$ is a trail that travels every edge of $G$.
An $Euler \ tour$ is an Euler trail such that the start and ending vertex are the same.

If we have an Eulerian trail, then at most $2$ vertices have an odd degree.

Theorem: if $G$ is connected and ever degree is even, then $G$ has an Eulerian tour.
If $G$ is connected and all but two degrees are even, then $G$ has an Eulerian trail.

Note that we might have an Eulerian tour when a graph is not connected,
for example, if one of the components is an isolated vertex while the other component has an Eulerian tour...

\section{Hamiltonian Cycles and Paths}
\subsection{Definitions}
A cycle such that every vertex is included exactly once is a $Hamiltonian \ cycle$,
whereas a path with the same quality is a $Hamiltonian \ path$.
In other words, the only way they differ is whether the end vertex is our start vertex.

A graph is $traceable$ if there exists a Hamiltonian path through it, while
a graph is $Hamiltonian$ if there is a Hamiltonian cycle.

\subsection{Tools}
If $G$ is Hamiltonian, then by deleting $k$ vertices, we obtain $\leq k$ components.
We note that this is necessary, not sufficient. Peterson's graph is a counterexample:
we can delete k vertices and obtain less than k components, but Peterson's graph
is not Hamiltonian.

\section{Planarity and Duality}
\subsection{Planarity}
A $plane \ graph$ is a graph drawn on the plane with no edge crossings.
A $planar \ graph$ is a graph that can be drawn on the plane.

$Euler's Formula$: for plane, connected graphs, \[e + 2 = r + n\]
$G$ is planar $<=>$ $G$ can be drawn on the surface of a sphere without crossings.

Thus we can make the outer region whichever region we want.

The skeletons of polyhedra are planar graphs, as well as 3-connected. 
This means that we can delete up to two of any vertices and still be connected.

\subsection{Duality}
We denote the dual of $G$ as $G*$.
If two regions have $k$ edges in common, then the corresponding vertices in $G*$ have $k$ edges between them as well.

\section{Colorings}
\subsection{Definitions}
\subparagraph{Vertex Colorings}
A $coloring \ of \ vertices$ is a function $c:H(V)\rightarrow\lbrace1,2,...,r\rbrace$ such that
if $(a,b) \in E(H)$, then $c(a) \neq c(b)$. 

The $chromatic \ number$ is the minimum number of colors $r$ such that
$r$ colors suffice, as in there exists a coloring with $r$ colors.
This is denoted as $X\left(G\right)$.

Another name for this complete subgraph is a $clique$.

The size of the max clique is the $clique \ size$: $\omega\left(G\right)$.

If we can find a complete subgraph, we have to have at least that number of vertices number of colors.
\[X\left(G\right) \geq \omega\left(G\right)\]

An $independent \ vertex \ set$ is a set of vertices with no edges between any of them.
We denote the maximum size of an independent set as $\alpha\left(G\right)$.

\subparagraph{Edge Colorings}
$X_e\left(G\right)$ or $X'\left(G\right)$ denotes the minimum number to color edges such that
a vertex has no two edges of the same color. This is the $edge \ chromatic \ number$ or the $chromatic \ index$.

The max degree of $G$ is denoted $\Delta(G)$.

An $independent \ edge \ set$ or a $matching$ is a set of edges with no vertices in common.
We denote the maximum size of a matching with $v(G)$.

\end {document}